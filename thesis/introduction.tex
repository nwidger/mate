\chapter{INTRODUCTION}

Nondeterministic execution is an unfortunate side effect of existing
multicore and multiprocessor systems.  Nondeterminism in parallel
programs can arise from many different factors such as differences in
thread scheduling, cache state and I/O delays.  Nondeterministic
execution leads to an array of problems:

\begin{itemize}
\item it causes programs to produce different output when executed
  with the same inputs

\item makes concurrency bugs such as race conditions hard to reproduce

\item creates bugs that lie dormant in multithreaded code for years
  before an input and thread interleaving are executed that causes
  them to be discovered

\item makes unit testing of multithreaded code difficult and less
  conclusive

\item causes major frustrations to programmers wishing to enjoy the
  benefits of parallel programming
\end{itemize}

However, recent research has shown that deterministic execution of
parallel programs is possible on today's hardware, and can be achieved
with overhead small enough that it is sufficient for debugging
purposes and, depending on the application, even for deployment.

A program is said to execute deterministically if it always produces
the same output given the same inputs.  Deterministic execution offers
several advantages such as more reliable testing, reproducible bugs
and more useful crash reports.  The overhead required to ensure
deterministic execution can be reduced if certain assumptions about
the program are made.  Strong determinism guarantees a deterministic
ordering of all shared memory accesses for a given input.  Weak
determinism guarantees a deterministic ordering of all lock
acquisitions for a given input.  Weak determinism yields deterministic
execution only if the program is data-race free, while strong
determinism makes no assumptions about program correctness.  Strong
determinism is concerned with all accesses to shared memory whereas
weak determinism requires that only lock operations are watched.
Because of this, the overhead to enforce strong determinism is higher
than weak determinism.

Most research in deterministic execution has been in the form of
run-time frameworks for C/C++ code that requires code to be recompiled
before it can be executed deterministically.  In contrast, my thesis
involved applying existing deterministic execution techniques to a
Java-like language that is executed in a virtual machine.  I believed
is that these techniques can be applied with lower overhead in a
virtual machine setting than in a compiled language.

%%% Local Variables: 
%%% mode: latex
%%% TeX-master: "thesis"
%%% End: 
