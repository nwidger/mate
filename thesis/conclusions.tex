\chapter{CONCLUSIONS}
\label{CONCLUSIONS}

\section{Conclusions}

The results gathered from the benchmarks do not back up my original
thesis: the results indicate that the overhead of implementing DMP in
a Java-like programming language such as maTe is not lower than a
C-like programming language.  I no longer think it likely that such an
implementation can beat out the performance of DMP implemented as a
runtime-library for compiled languages such as C/C++.

However, I still believe that implementing DMP inside a virtual
machine is still a sound idea.  On all but the most extreme DMP
parameter settings, the performance overhead seen in the benchmarks
was acceptable.  This is especially true if running the benchmarks
would uncover a bug in the implementation.  Furthermore, there are
definitely advantages in not having to recompile a maTe program in
order to run with/without DMP enabled.

I also conclude that implementing an efficient multithreaded virtual
machine is itself a difficult task, especially when starting with a
virtual machine that was initially designed to be only single
threaded.  Reducing thread contention when accessing global data
structures, especially the heap, was a real pain point.

Analyzing the results was made more difficult due to the fact that in
many of the benchmarks the maTe virtual machine's performance worsened
as more threads were added.

It is also clear from the results that using different ownership table
depths was not particularly effective in reducing the amount of time
threads block due to ownership changes on most benchmarks.

I definitely found that the Table class, being the most complex native
class and also the only built-in data structure, was one of the most
difficult parts of the VM to get working with DMP.

\section{Future Work}

Future work could include modifying the ownership table policy to
include an adaptive algorithm that attempts to learn the shared memory
profile of the maTe program and adjust the ownership of objects in an
atttempt to increase the likelihood that a thread can access a given
object without needing to block.

Future work could also include making the maTe virtual machine more
efficient when run with many threads.

As stated earlier, the current implementation runs the serial garbage
collector at the end of serial mode when the heap is at least $90\%$
full.  There may be more efficient ways to do this.

Reimplement the virtual machine's heap to allocate large blocks of
memory up front instead of calling $malloc$ for every allocation of a
virtual machine object.

%%% Local Variables: 
%%% mode: latex
%%% TeX-master: "thesis"
%%% End: 
