\chapter{CONCLUSIONS}
\label{CONCLUSIONS}

\section{Conclusions}

The results gathered from the benchmarks do not back up my original
thesis.  The average overhead of the DMP enabled benchmarks was
between $19\%$ and $2800\%$.  These results indicate that the overhead
of implementing DMP in a Java-like programming language such as maTe
is not lower than a C-like programming language.  I no longer think it
likely that such an implementation can beat out the performance of DMP
implemented as a runtime-library for compiled languages such as C/C++.

However, I still believe that implementing DMP inside a virtual
machine is a sound idea.  On all but the most extreme DMP parameter
settings, the performance overhead seen in the benchmarks was
acceptable.  On longer-running benchmarks, the acceptability may not
hold.  Furthermore, there are definitely advantages to not having to
recompile a maTe program in order to run with/without DMP enabled.
Being able to quickly run the same program repeatedly using different
DMP parameters offers the possibility of ``fuzz testing'' a program in
addition to making it much easier to find optimal parameters for a
given benchmark and input data.

I also conclude that implementing an efficient multithreaded virtual
machine is itself a difficult task, especially when starting with a
virtual machine and language that was initially designed to be single
threaded.  Reducing thread contention when accessing global data
structures, especially the heap, was difficult.  Analyzing the results
was made more difficult due to the fact that in many of the benchmarks
the maTe virtual machine's performance worsened as more threads were
added.

It is also clear from the results that using different ownership table
depths was not particularly effective in reducing the amount of time
threads block due to ownership changes on most benchmarks.  There are
a number of possibilities for this result.  One is that the particular
set of benchmarks chosen did not benefit from different depths.
Another possibility is that because most of the benchmarks stored
their data in $Table$ instances shared across threads, using any depth
greater than one would mean every ownership change to a $Table$
instance would cause all the key/value pairs objects in the $Table$ to
change owner as well.  In a benchmark that stores their data in a
handful of global $Tables$, this would essentially serialize access to
the table.  In general, I found that the $Table$ class, being the most
complex native class and also the only built-in data structure, was
one of the most difficult parts of the virtual machine to get working
with DMP.

\section{Future Work}

Future work could include modifying the ownership table policy to
include an adaptive algorithm that attempts to learn the shared memory
access patterns of a maTe program and preemptively change the
ownership of an object in an attempt to increase the likelihood that a
thread can access a given object without the need to block.

Future work could also include making the maTe virtual machine more
efficient when run with many threads.  This could include
reimplementing the virtual machine's heap to allocate large blocks of
memory up front instead of calling $malloc$ for every allocation of a
virtual machine object.

As stated earlier, the current implementation runs the serial garbage
collector at the end of serial mode when the heap is at least $90\%$
full.  There may be better ways to integrate the garbage collector
with DMP.  Doing this while maintaining DMP's guarantee of
deterministic execution could be difficult.

Another possibility would be modifying the maTe language to allow
annotating the source code to help the virtual machine choose better
DMP parameters.  For example, if the programmer can guarantee that a
data structure is only accessed by a single thread, an annotation
could indicate such and DMP checks on that object could be skipped.
Static analysis could also be used to determine when such
optimizations are possible.

There are plenty of opportunities to optimize the maTe compiler.
Improving the maTe compiler so that object fields are not
unnecessarily accessed when they already exist in the frame's local
variable array or operand stack could greatly reduce the amount of
blocking performed by a thread and thus improve performance.

Finally, DMP could be implemented in a more mature Java-like
multithreaded virtual machine such as the official HotSpot Java
virtual machine.  Poor multithreaded performance in the maTe virtual
machine made conclusions about DMP in a Java-like language difficult.
Adding DMP to a well-tuned multithreaded virtual machine could make
such statements more conclusive.

%%% Local Variables: 
%%% mode: latex
%%% TeX-master: "thesis"
%%% End: 
