\chapter{IMPLEMENTATION}
\label{IMPLEMENTATION}

\section{maTe Language}

Originally, the maTe language currently only allowed for integer
arithmetic using the predefined Integer class.  I added a new
predefined class, Real, that wraps a float primitive to allow
scientific applications that require floating point calculations to be
written.

In order to test multithreaded programs, the maTe language needed to
be modified in order to allow user-created threads.  I chose to use
Java's threading model by adding a new predefined $Thread$ class.
Users extend this class and override the $run$ method that is called
after the user begins execution of the thread with a call to its
$start$ method.

In order to allow for critical sections and synchronization, a monitor
in the form of a mutex was added to each object in the virtual machine
that can be acquired through new $monitorenter$ and $monitorexit$
instructions.  The language was extended to allow for synchronized
blocks, the body of which is executed only after acquiring the monitor
of a particular object.  Three new methods were added to the base
Object class: $wait$, $notify$ and $notifyAll$ adding support for
asynchronous events.  These new methods function entirely similarly to
those in Java.  A thread can block for another thread to terminate by
calling the $join$ method on a particular $Thread$ instance or block
for a specific length of time using the $sleep$ method.

These changes to the language required changes to all of the
development tools.  The grammar of the language used by the compiler
was be modified to include synchronized blocks.  At code generation
time, the compiler emits the new $monitorenter$ and $monitorexit$
instructions at the entrance and exit of each synchronized block.
Corner cases involving $break$ or $return$ statements inside nested
synchronized blocks must be accounted for.  To do so, the compiler
will need to maintain a monitor stack to ensure that all currently
acquired monitors are released regardless of the execution path.  The
assembler will be modified to be aware of the new instructions and
their respective opcodes.

\section{maTe Virtual Machine}

The majority of the changes required to test this thesis will be in
the virtual machine.  The first step is supporting multiple user
threads in the virtual machine.  Each $Thread$ instance will be
allocated a virtual machine stack and will be executed using the
pthreads threading library.  Further synchronization of virtual
machine data structures such as the heap may be necessary.  Because a
$Thread$ instance cannot be collected while its $run$ method is still
executing, regardless of its accessibility via the global object
graph, $Thread$ instances will need to be protected from the garbage
collector by adding them to a global thread set whose contents are not
considered for deletion by the garbage collector.  The garbage
collector will also use this set to iterate over the stack of each
thread while marking the roots during its mark phase.  Finally, the
virtual machine must be modified to include a mutex lock in the
implementation of each object and must implement the new
$monitorenter$ and $monitorexit$ instructions.

Secondly, DMP must be implemented inside the virtual machine.  The
ownership table will not be stored as a global data structure but
instead distributed amongst all objects: each object will store an
$owner$ field.  If $owner$ is $null$, the object is shared.  If
$owner$ is not $null$, the object is private and $owner$ specifies the
thread ID of its current owner.  The fetch/execute loop of each thread
will keep a executed instructions count for tracking quantum size.
The implementation of the $getfield$ and $putfield$ instructions will
be modified to follow the ownership table policy from the DMP paper
when checking the $owner$ field before allowing the instruction to
execute.  Two global barriers will be allocated when the virtual
machine initializes.  Calls to the first barrier will be placed in
spots where a thread must block for serial mode:

\begin{itemize}
\item upon executing a communicating $getfield$ or $putfield$
  instruction
\item upon creation of a $Thread$ instance
\item upon termination of the $run$ method of a $Thread$ instance
\item upon reaching the end of its quantum in the fetch/execute loop
\end{itemize}

The heap will store a set containing all $Thread$ references sorted by
creation time.  This will be used to wake each thread in a
deterministic order so that it can execute its serial segment.  Upon
reaching the end of its serial segment, each thread will call the
second global barrier.  After all threads have reached the second
barrier, a new round begins with all threads executing in parallel.
